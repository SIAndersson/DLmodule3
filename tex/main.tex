\documentclass{article}
\usepackage{graphicx} % Required for inserting images
\usepackage[numbers]{natbib} % Numeric references
\usepackage{hyperref}

\hypersetup{
    colorlinks=true, % Enable colored links
    linkcolor=black, % Color for internal links (sections, pages)
    citecolor=black, % Color for citation links
    urlcolor=black, % Color for external links (URLs)
    filecolor=black % Color for file links
}

\title{Blank}
\author{Sebastian Andersson \and Lo Heander \and Andreas Bexell}
\date{June 2025}

\begin{document}

\maketitle

\section{Introduction}

We choose the Flow matching and Diffusion model families. 

\section{Model Families}

\subsection{Flow matching}

\subsection{Diffusion}

Diffusion models is a type of deep generative models that has found many applications recently, especially for image generation. The models are based on two diffusion stages. In the \emph{forward diffusion stage}, normally distributed noise is incrementally added to the training data, until the original data is entirely obscured by noise. In the next stage, \emph{reverse diffusion stage}, the model is trained to recover the original data from the noisy data at each step. Ultimately, this enables the model to generate new data from pure noise. If conditioned with an additional label or prompt, that can be used to enable generation of data from a combination of noise and instruction.

Croitoru et al.~\cite{croitoru2023diffusion} conducts a survey on how diffision models apply to computer vision use-cases.

\paragraph{Denoising diffusion probabalistic models} Adds gaussian noise in many, typically more than 1000 discreete steps. The process is to train a neural network to remove noise in each step.

\paragraph{Score-based generative models} estimates a score function $\nabla_x \log p_t(x)$ at each noise level to determine the direction of ``less noise''. The model trains to minimize the loss over several noise levels, and uses the score function to to simulate differential equations that can apply the noise in reverse, that is, to remove the noise.

\paragraph{Stochastic differential equations} is a continuous time model combining elements from DDPMs and 

\bibliographystyle{plainnat}
\bibliography{references}

\end{document}
